\chapter{Conclusiones y trabajo futuro}
\textit{Las buenas prácticas de la Ingeniería de Software son esenciales para el desarrollo de
productos software de gran envergadura, donde un grupo de ingenieros trabajan en equipo de cara a desarrollar la solución. \cite{mall2018fundamentals}
}

\section{Relación del trabajo relacionado con los estudios cursados}
De forma evidente, para la realización del presente trabajo, han resultado imprescindibles los conocimientos adquiridos del
Grado de Ingeniería Informática, sobre todo aquellos propios de las asignaturas de la rama de \textit{Ingeniería de Software}.
A continuación se realizará un repaso de la influencia de estos estudios en el trabajo:

Para la elección entre todo el abanico de metodologías \textit{software} disponibles, una metodología acorde al proyecto, ha sido
necesario el estudio de las mismas. La asignatura \textit{Proceso de software} ha facilitado el proceso de explorar
cada una de las metodologías, contemplando sus limitaciones e impacto en el proyecto. Tomando finalmente la metodología \textit{PXP}
como guía fundamental.

Gracias  a la asignatura de \textit{Análisis y Estudio de Requisitos}, se han identificado y analizado todos
los requisitos a incorporar en el aplicativo, todos ellos especificados gracias al estándar IEEE Std 830-1998, visto previamente
en la asignatura.

Obtener una cobertura de código realmente extensa ha sido realmente importante de cara al uso de la
metodología \textit{PXP}. Esto ha sido posible por el seguimiento de la metodología \textit{TDD}
junto las bases y conocimientos de  la asignatura \textit{Análisis Validación y Depuración de Software}.

El uso y comprendimiento de patrones de diseño y principios de \textit{código limpio} reflejados en la materia \textit{Diseño de Software}
han propiciado el uso en el aplicativo de \textit{manejadores de estados}, establecer una arquitectura acorde
al proyecto y tener un código más \textit{modular}, legible y funcional.

Por último, han sido necesarias una serie de competencias transversales que han sido estudiadas
y vistas durante todo el Grado. Aquellas más remarcables y presentes en este proyecto son las que siguen:

\begin{itemize}
    \item[$\bullet$] \textbf{Análisis y resolución de problemas}: una de las
    más recurrentes y presentes durante la realización de este trabajo. Gracias a esta, se han
    analizado la problemática a resolver, establecido y usado un plan de acción y posteriormente
    verificado su correcto funcionamiento.
    \item[$\bullet$] \textbf{Instrumental específica}: Gracias a esta competencia
    se han analizado los principales \textit{frameworks} de desarrollo móvil actuales. 
    Eligiendo \textit{Flutter} como el entorno de desarrollo principal y empleando de forma
    análoga la plataforma \textit{back\-end} \textit{Firebase}.
    \item[$\bullet$] \textbf{Instrumental específica}: Gracias a esta competencia
    se han analizado los principales \textit{frameworks} de desarrollo móvil actuales. 
    Eligiendo \textit{Flutter} como el entorno de desarrollo principal y empleando de forma
    análoga la plataforma \textit{back\-end} \textit{Firebase}.
    \item[$\bullet$] \textbf{Diseño y proyecto}: Al tratarse de un proyecto individual
    ha sido realmente necesaria esta competencia, contemplando y anticipando
    posibles limitaciones de recursos y en general. El resultante de este proceso
    ha sido la creación de un producto propio a medida.
\end{itemize}

\section{Conclusiones}
Como conclusiones finales, haciendo una retrospectiva de los objetivos
marcados en el inicio del proyecto, se puede afirmar que se ha cumplido
 el objetivo primordial: Desarrollar un \textit{MPV} de un aplicativo móvil capaz de:
i) adaptar la experiencia de resolución de un nonograma en un medio digital
ii) crear una funcionalidad capaz de crear tus propios \textit{nonogramas} y
compartirlos con la comunidad.

Para ello, se ha contemplado y aplicado finalmente la opción de extender el uso 
del aplicativo en las dos sistemas operativos móviles principales: \textit{iOS}
y \textit{Android}, empleando el \textit{framework} de desarrollo móvil
multiplataforma \textit{Flutter}. Plataforma que, durante el desarrollo de
este proyecto, ha ido adoptando nuevas actualizaciones y versiones que han
facilitado y mejorado el desarrollo del aplicativo. Una de las
principales actualizaciones fue la versión en entorno \textit{stable} \textit{Flutter 2.0.0}, 
la cual hizo necesario adaptar la característica de \textit{null-safety} en el proyecto,
actualmente una vital característica en proyectos \textit{Flutter}. Es notable 
destacar que esta migración, tuvo que realizarse hasta que todos las librerías de terceros
existentes del proyecto adoptaran la propiedad \textit{null-safety}. Este hecho hizo
requerir más tiempo en completar e implementar algunas funcionalidades.

Se ha ahondado y estudiado el campo de los servicios \textit{en nube}, barajando
varias soluciones disponibles actuales, como entornos: \textit{Azure}, \textit{AWS}, ... empleando finalmente la plataforma de datos
de tipo \textit{real-time} de \textit{Firebase} (\textit{Cloud Firestore}), junto a sus principales funcionalidades
tales como: \textit{Firebase Authentication} y \textit{Analytics}. A diferencia
del uso de \textit{Flutter}, para el empleo de estas funcionalidades fue
necesario su estudio, ya que no se tenía una experiencia profesional. Sin embargo,
fue una decisión acertada ya que su combinación junto con \textit{Flutter} es una 
de las más recurrentes en este ámbito.

No solo se han aplicado los principios y las bases marcadas por la arquitectura hexagonal
\textit{Clean Architecture}, sino se ha complementado junto a la metodología
de tests automatizados \textit{TDD}, un imprescindible en la tarea de
verificar y validar el sistema de principio a fin.

Toda este serie de apartados han podido cumplirse por el análisis de una gran
cantidad de artículos, medios digitales y libros académicos. Este gran abanico
de contenido ha facilitado notablemente el desarrollo final del proyecto,
aplicando patrones de diseño, buenas praxis y uso del famoso compendio de \textit{clean code}.

Por último y no menos importante, desde la visión de un aficionado de los \textit{nonogramas}, ha sido necesario profundizar en el mundo
de los rompecabezas, comprender más a fondo su lógica y complejidad, más allá de su resolución, contemplando
y recopilando características de otras soluciones existentes.

\section{Trabajo futuro}
La realización de este trabajo se han centrado, mayoritariamente, en el cumplimiento de los objetivos
anteriormente nombrados. Sin embargo, durante todo el proceso, se han encontrado una serie de puntos interesantes
a incluir.

La gran actualización de \textit{Flutter 2.0.0}, ha permitido desde su fecha de salida,
compilar proyectos desarrollados con \textit{Flutter} en entornos \textit{web} y
sistemas operativos de escritorio tales como: \textit{Windows}, \textit{Linux} y \textit{MacOS}.
No obstante, su proceso no es inmediato, requiere de adaptar la mayoría de componentes
visuales (\textit{widgets}) a las proporciones de los monitores actuales, además de, adaptar
una gran cantidad de librerías externas de este proyecto en estos entornos.
Ya que actualmente, en este \textit{framework}, los desarrollos en \textit{web} son los más recurrentes después de los móviles,
sería interesante adaptar este aplicativo a este medio digital. Posteriormente, cuando estén
más normalizados los desarrollos en aplicaciones de escritorio, empezar a adaptar el aplicativo en estos
entornos.

De forma análoga, el paquete \textit{Google Mobile Ads} fue impulsado tras esta actualización,
este servicio nos permite monetizar el aplicativo mediante anuncios. Esta característica
podría introducirse de forma no intrusiva en la aplicación y puede convertirse en una de las vías principales para generar ingresos sin
llegar a entorpecer al usuario, siempre pensando en la experiencia de juego.

Finalmente, aunque el aplicativo tiene una temática visual bien diferenciada y
definida, presenta numerables limitaciones de diseño. Incluir más animaciones, diferentes \textit{assets} visuales y
colecciones de iconos personalizados, mejoraría notablemente la estética del aplicativo, un requisito
fundamental hoy en día en aplicaciones de su misma categoría.