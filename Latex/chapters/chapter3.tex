\chapter{Análisis del problema}
\textit{La Ingeniería de Requisitos (IR) es el área más importante de la Ingeniería de Software y posiblemente de todo el 
ciclo de vida de una solución software (SDLC)~\cite{chakraboty2012requirements}. Esta etapa es la responsable de que los 
requisitos recién detectados, aún incompletos e imprecisos, se transformen en especificaciones formales del aplicativo final.}

\section{Especificación de requisitos}
Para llevar a cabo la especificación total de requisitos se seguirá la norma tradicional establecida por el estándar internacional 
IEEE Std 830-1998~\cite{8559686}, elegido por una gran mayoría de jefes de departamentos software por su gran agilidad en la fase de gestión 
de requisitos~\cite{guzman2018impacto}.

A continuación, de acuerdo a la normativa ISO elegida, se mostrarán los contenidos acompañados por diagramas y buenas prácticas.

\subsection{Propósito}
El propósito de esta sección es la de definir y formalizar los requerimientos que debe incorporar el \textit{MVP} del aplicativo, facilitando
y guiando el desarrollo del mismo.

\subsection{Ámbito}
El ámbito, como se ha comentado en capítulos anteriores, es el de las aplicaciones móviles disponibles en plataformas \textit{iOS} y 
\textit{Android}.

El aplicativo en su versión \textit{MVP} adoptará el nombre provisional de \textit{NonoChallenge}, compuesto por el juego de palabras:
 \textit{Nonograma} junto con el término anglosajón \textit{Challenge} (reto).
En el cual, el usuario hará uso de sus servicios propios y \textit{en nube} tales como inicios de sesión, interacción con base de datos y
sincronización.

\subsection{Terminología}
Los términos relacionados con la ontología del sistema se ven enumerados y descritos por la Tabla~\ref{fig:table2}.

\begin{table}[H]
  \caption{Glosario de términos ontológicos del aplicativo}
    \begin{tabular}{ | c | c |}
      \hline
      \thead{Término} & \thead{Descripción} \\
      \hline
      \makecell{Usuario} &  Persona que hará el uso del conjunto de funcionalidades del aplicativo final  \\
      \hline
      \makecell{Nonograma} &  \makecell{Rompecabezas de MxN dimensiones}  \\
      \hline
      \makecell{Nivel} &  Nonograma a crear o resolver por el Usuario\\
      \hline
      \makecell{Progreso} &  \makecell{Datos relacionados con la persistencia de la resolución de un nivel} \\
      \hline
    \end{tabular}
    \label{fig:table2}
\end{table}

Por otra parte, los términos técnicos que componen el sistema son los que siguen:

\begin{table}[H]
    \caption{Glosario de términos técnicos del aplicativo}
      \begin{tabular}{ | c | c |}
        \hline
        \thead{Término} & \thead{Descripción} \\
        \hline
        \makecell{Nombre \\ de usuario} &  \makecell{Nombre que adoptará el Usuario de forma opcional para su\\identificación en el aplicativo.}  \\
        \hline
        \makecell{Email} &  \makecell{Cuenta de correo que hará uso el usuario para acceder a los servicios \textit{en nube}}  \\
        \hline
        \makecell{Fecha de \\ publicación} &  Fecha en la que el usuario crea y publica un nivel  \\
        \hline
        \makecell{Figura} &  Nombre identificativo de un nonograma \\
        \hline
        \makecell{Vidas} &  Número de intentos en la resolución de un nivel \\
        \hline
        
      \end{tabular}
      \label{fig:table3}
  \end{table}

  \subsection{Modelo de Dominio}
  Una vez introducida la términología propia del sistema, para un mayor entendimiento del contexto del mismo, se representa un  
  diagrama de clases de acuerdo a las reglas clásicas UML, como se pude visualizar en la Figura~\ref{fig:dom1}.

  \begin{figure}[H]
    \centering
  \begin{tikzpicture}
    \begin{umlpackage}[fill=black!10]{Sistema}
      \umlclass[x=1,y=1]{Usuario}{uid : string \\ nombreUsuario : string? \\ email : string? \\ numCompletados : int}{}
      \umlclass[x=2,y=-3.8]{Nivel}{uid : string \\ fechaPub : datetime? \\ figura : string \\ vidas : int}{}
      \umlclass[x=9,y=-3.8]{Nonograma}{  width : int \\ height : int \\ celdasCorrectas : List<int>}{}
      \umlclass[x=8,y=1]{<<data>> Progreso}{id : int \\ completado : int \\ celdasMarcadas : string? \\ celdasDescartadas : string? \\ vidas : int}{}
    \end{umlpackage}
      
    \umlassoc[pos=0.6,arg=crea,mult1=1,mult2=0*, anchors=-130 and 130] {Usuario}{Nivel}
    \umlassoc[pos=0.6,arg=resuelve,mult1=1,mult2=0*, anchors=-60 and 55] {Usuario}{Nivel}
    \umlassoc[pos=0.7,arg=contiene,mult1=1,mult2=1] {Nivel}{Nonograma}
    \umlassoc[arg=tiene,mult1=1,mult2=0..1] {Nivel}{<<data>> Progreso}
    % \umlassoc[geometry=-|-, arg1=tata, mult1=*, pos1=0.3, arg2=toto, mult2=1, pos2=2.9, align2=left]{C}{B}
    % \umlunicompo[geometry=-|, arg=titi, mult=*, pos=1.7, stereo=vector]{D}{C}
    % \umlaggreg[arg=tutu, mult=1, pos=0.8, angle1=30, angle2=60, loopsize=2cm]{D}{D}
    % \umlinherit[geometry=-|]{D}{B}
    \end{tikzpicture}
    \caption{Diagrama del modelo de dominio}
   \label{fig:dom1}
  \end{figure}

  A continuación, se comenta la función de cada de una de las clases conceptuales y se enumeran en las Tablas~\ref{fig:table3}-~\ref{fig:table6} 
  los atributos que las componen. \textit{(La anotación ? al lado del tipo refleja su posibilidad de
  nulidad en el sistema.)}
  

  \textbf{$\bullet$ Clase Usuario}: Conjunto de datos del usuario relacionados con el sistema.

  \begin{table}[H]
    \centering
    \caption{Atributos de la clase Usuario}
      \begin{tabular}{ | c | c |}
        \hline
        \thead{Atributos de Usuario} & \thead{Descripción} \\
        \hline
        \makecell{uid\\\textit{\textit{string}}} & \makecell{Cadena de caracteres único que identifica al usuario a \\nivel interno}\\
        \hline
        \makecell{nombre de Usuario\\\textit{string?}} & Pseudónimo único a elegir por el usuario en el sistema \\
        \hline
        \makecell{email\\\textit{string?}} &  Cuenta de correo de registro única para el usuario\\
        \hline
        \makecell{numCompletados\\\textit{int}} &  \makecell{Número de niveles completados por el usuario} \\
        \hline
      \end{tabular}
      \label{fig:table3}
  \end{table}

  \textbf{$\bullet$ Clase Nivel}: Muestra las características de un determinado nivel a resolver.

  \begin{table}[H]
    \centering
    \caption{Atributos de la clase Nivel}
      \begin{tabular}{ | c | c |}
        \hline
        \thead{Atributos de Nivel} & \thead{Descripción} \\
        \hline
        \makecell{uid\\\textit{\textit{string}}} & \makecell{Cadena de caracteres único que identifica al nivel}\\
        \hline
        \makecell{fecha de Publicación\\\textit{dateTime?}} & Fecha de creación del nonograma en caso de llegar a publicarse \\
        \hline
        \makecell{figura\\\textit{string}} &  Nombre del nivel mostrado una vez resuelto\\
        \hline
        \makecell{vidas\\\textit{int}} &  \makecell{Número de intentos que dispone un usuario para la resolución de \\un nivel} \\
        \hline
      \end{tabular}
      \label{fig:table4}
  \end{table}

  \textbf{$\bullet$ Clase Progreso}: Clase de persistencia que contiene los datos durante la resolución de un nivel.

  \begin{table}[H]
    \centering
    \caption{Atributos de la clase Progreso}
      \begin{tabular}{ | c | c |}
        \hline
        \thead{Atributos de Progreso} & \thead{Descripción} \\
        \hline
        \makecell{id\\\textit{int}} & \makecell{Número de identificación del progreso de un determinado nivel}\\
        \hline
        \makecell{completado\\\textit{int}} & \makecell{Indica con un 1 si el nivel está resuelto y 0 si no lo está\\
        (simulando un dato de tipo booleano)}\\
        \hline
        \makecell{celdasMarcadas\\\textit{string}} &  \makecell{Números de las celdas pintadas separados por delimitadores\\
        (simulando un dato de tipo lista de enteros)}\\
        \hline
        \makecell{celdasDescartadas\\\textit{string}} &  \makecell{Números de las celdas descartadas separados por delimitadores\\
        (simulando un dato de tipo lista de enteros)}\\
        \hline
        \makecell{vidas\\\textit{int}} &  \makecell{Números de vidas que dispone el usuario en ese momento}\\
        \hline
      \end{tabular}
      \label{fig:table5}
  \end{table}
  \hfill 

  \textbf{$\bullet$ Clase Nonograma}: Contiene los datos del nonograma de un determinado nivel.

  \begin{table}[H]
    \centering
    \caption{Atributos de la clase Nonograma}
      \begin{tabular}{ | c | c |}
        \hline
        \thead{Atributos de Nonograma} & \thead{Descripción} \\
        \hline
        \makecell{width\\\textit{int}} & \makecell{Número de filas de celdas que compone el Nonograma}\\
        \hline
        \makecell{height\\\textit{int}} & Número de columnas de celdas que compone el Nonograma \\
        \hline
        \makecell{celdasCorrectas\\\textit{List<int>}} & \makecell{Lista de los números de celdas que componen \\
        la solución del Nonograma} \\
        \hline
      \end{tabular}
      \label{fig:table6}
  \end{table}

  \subsection{Límites del Sistema}
  Para representar los límites de la aplicación, se emplea un diagrama de contexto en el que se muestra la jerarquía de los actores 
  que van a interactúan con el aplicativo.

\tikzumlset{fill usecase=white}
\begin{figure}[H]
  \centering
\begin{tikzpicture}
    \begin{umlsystem}[x=5,y=-3,fill=black!10] {Aplicativo} % empty title
        % \umlusecase[name=a,width=1.5cm] {Use case}
        % \umlusecase[name=b,x=6,width=2.5cm] {Use case b}
        % \umlusecase[name=c,x=6,y=-3,width=2.5cm] {Use case c}
        % \umlusecase[name=d,y=-3,width=2.5cm] {Use case d}
    \end{umlsystem}


    \umlactor[y=-1] {Usuario}
    \umlactor[y=-4] {Usuario Registrado}

    \umlassoc{Usuario}{Aplicativo}
    \umlinherit{Usuario Registrado}{Usuario}
    
    % \umlextend{a}{b}
    % \umlinclude{c}{d}

    % manual versions of the above
    % \draw [tikzuml dependency style] (a) -- node[above] {$\ll \text{extend} \gg$} (b);
    % \draw [tikzuml dependency style] (d) -- node[above] {$\ll \text{include} \gg$} (c);

    % bent association
  \end{tikzpicture}
  \caption{Diagrama de contexto del aplicativo}
   \label{fig:dom2}
\end{figure}

Como se puede apreciar en el diagrama de la Figura~\ref{fig:dom2}, únicamente van a interacutar dos actores con el sistema: i) el usuario sin registro,
que emplea únicamente las funciones locales del sistema y ii) el usuario registrado, que podrá emplear tanto las funciones locales como
\textit{en nube} del aplicativo.

\subsection{Restricciones del Sistema}

En este apartado se muestran algunas de las restricciones que pueden impedir algunas de las funcionalidades que ofrece el sistema
al usuario, vistos en la Tabla ~\ref{fig:table7}.

\begin{table}[H]
  \centering
  \caption{Restricciones del sistema}
    \begin{tabular}{ | c | c |}
      \hline
      \thead{Restricción} & \thead{Descripción} \\
      \hline
      \makecell{Memoria del\\ dispositivo} & \makecell{Posibilidad de que el dispositivo no tenga suficiente memoria
      \\disponible para guardar el progreso de la resolución de los niveles.\\\textit{Probabilidad prácticamente despreciable} }\\
      \hline
      \makecell{Conexión a\\internet} & \makecell{Las funcionalidades \textit{en nube} requieren de conexión a internet\\ para poder 
      llegar a usarse.} \\
      \hline
      \makecell{Cuenta\\de Google} &  \makecell{El usuario pueder carecer de una cuenta de correo de Google\\necesaria para hacer uso de las funcionalidades 
      \textit{en nube}.} \\
      \hline
    \end{tabular}
    \label{fig:table7}
\end{table}

\subsection{Características del Sistema}
Una vez vistas las limitaciones y restricciones del sistema, se pueden analizar las características finales 
a incorporar en la aplicación, mediante \textit{casos de uso}.

\subsubsection{Casos de uso principales}

Estos, son los que se encontrarán en la \textit{Pantalla Principal} y \textit{Pantalla de Seleción de Nonogramas} 
del aplicativo.

\begin{figure}[H]
  \centering
\begin{tikzpicture}
  \begin{umlsystem}[x=5, fill=black!10]{Aplicativo} 
    \umlusecase[name = usecase1, width=2cm]{Jugar nivel}
    \umlusecase[name = usecase2, x=6,width=2cm]{Seleccionar nivel}
    \umlusecase[name = usecase3, x=4, y=-2, width=2cm]{Acceder Tutorial} 
    \umlusecase[name = usecase4, y=-5]{Acceder a Online}
    \umlusecase[name = usecase5, x=6, y=-4, width=2cm]{Publicar nonograma}
    \umlusecase[name = usecase6, x=6, y=-6, width=2cm]{Resolver nivel online}
    \umlusecase[name = usecase7, y=-3]{Acceder a Ajustes}
    \end{umlsystem}
    \umlactor[y=-1]{Usuario}
    \umlactor[y=-4]{Usuario Registrado} 
    \umlinherit{Usuario Registrado}{Usuario}

    \umlinclude[]{usecase1}{usecase2}
    \umlassoc[]{Usuario}{usecase1}
    \umlassoc[]{Usuario Registrado}{usecase4}
    \umlinclude[]{usecase4}{usecase5}
    \umlinclude[]{usecase4}{usecase6}
    \umlassoc[]{Usuario}{usecase3}
    \umlassoc[]{Usuario}{usecase7}

\end{tikzpicture}
\caption{Diagrama de casos de usos principales}
\label{fig:caso1}
\end{figure}

\subsubsection{Casos de uso de resolución de nivel}

Estarán disponibles en la pantalla de resolución de nivel.

\begin{figure}[H]
  \centering
\begin{tikzpicture}
  \begin{umlsystem}[x=5, fill=black!10]{Aplicativo} 
    \umlusecase[name = usecase4, y=-4]{Jugar nonograma}
    \umlusecase[name = usecase5, x=7, y=-3, width=2cm]{Superar nivel}
    \umlusecase[name = usecase6, x=4, y=-6]{Perder nivel}
    \umlusecase[name = usecase7, y=-6, width=1.5cm]{Cargar progreso}
    \umlusecase[name = usecase8, y=-2, width=1.5cm]{Salir del nivel}
    \umlusecase[name = usecase9, y=-2, x=5, width=1.5cm]{Guardar progreso}
    \end{umlsystem}
    \umlactor[y=-4]{Usuario} 

    \umlassoc[]{Usuario Registrado}{usecase4}
    \umlVHextend[]{usecase5}{usecase4}
    \umlVHextend[]{usecase6}{usecase4}
    \umlinclude[]{usecase4}{usecase7}
    \umlinclude[]{usecase4}{usecase8}
    \umlVHextend[]{usecase9}{usecase8}

\end{tikzpicture}
\caption{Diagrama de casos de usos en pantalla de resolución de nivel}
\label{fig:caso2}
\end{figure}

\subsubsection{Casos de uso de ajustes}

Casos de uso visibles en la \textit{Pantalla de Ajustes} del sistema.

\begin{figure}[H]
  \centering
\begin{tikzpicture}
  \begin{umlsystem}[x=5, fill=black!10]{Aplicativo} 
    \umlusecase[name = usecase1]{Cambiar tema}
    \umlusecase[name = usecase3, x=4, y=-1, width=2cm]{Seleccionar vidas} 
    \umlusecase[name = usecase4, x=1, y=-6]{Cerrar sesión}
    \umlusecase[name = usecase5, x=3, y=-5]{Sincronizar datos}
    \umlusecase[name = usecase6, x=4, y=-3, width= 2.5cm]{Seleccionar auto-guardado}
    \umlusecase[name = usecase7, y=-3.5]{Borrar datos}
    \end{umlsystem}
    \umlactor[y=-1]{Usuario}
    \umlactor[y=-4]{Usuario Registrado} 
    \umlinherit{Usuario Registrado}{Usuario}

    \umlassoc[]{Usuario}{usecase1}
    \umlassoc[]{Usuario Registrado}{usecase4}
    \umlassoc[]{Usuario Registrado}{usecase5}
    \umlassoc[]{Usuario}{usecase3}
    \umlassoc[]{Usuario}{usecase6}
    \umlassoc[]{Usuario}{usecase7}

\end{tikzpicture}
\caption{Diagrama de casos de ajustes}
\label{fig:caso1}
\end{figure}
