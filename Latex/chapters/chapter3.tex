\chapter{Análisis del problema}
\textit{``La Ingeniería de Requisitos (IR) es el área más importante de la Ingeniería de Software y posiblemente de todo el 
ciclo de vida de una solución software (SDLC)''~\cite{chakraboty2012requirements}. Esta etapa es la responsable de que los 
requisitos recién detectados, aún incompletos e imprecisos, se transformen en especificaciones formales del aplicativo final.}

\section{Especificación de requisitos}
Para llevar a cabo la especificación total de requisitos se seguirá la norma tradicional establecida por el estándar internacional 
IEEE Std 830-1998~\cite{8559686}, elegido por una gran mayoría de jefes de departamentos software por su gran agilidad en la fase de gestión 
de requisitos~\cite{guzman2018impacto}.

A continuación, de acuerdo a la normativa ISO elegida, se mostrarán los contenidos acompañados por diagramas y buenas prácticas.

\subsection{Propósito}
El propósito de esta sección es la de definir y formalizar los requerimientos que debe incorporar el \textit{MVP} del aplicativo, facilitando
y guiando el desarrollo del mismo.

\subsection{Ámbito}
El ámbito, como se ha comentado en capítulos anteriores, es el de las aplicaciones móviles disponibles en plataformas \textit{iOS} y 
\textit{Android}.

El aplicativo en su versión \textit{MVP} adoptará el nombre provisional de \textit{NonoChallenge}, compuesto por el juego de palabras:
 \textit{Nonograma} junto con el término anglosajón \textit{Challenge} (reto).
En el cual, el usuario hará uso de sus servicios propios y \textit{en nube} tales como inicios de sesión, interacción con base de datos y
sincronización.

\subsection{Terminología}
Los términos relacionados con la ontología del sistema se ven enumerados y descritos por la \autoref{fig:glo2}.

\begin{table}[H]
  \caption{Glosario de términos ontológicos del aplicativo}
    \begin{tabular}{ | c | c |}
      \hline
      \thead{Término} & \thead{Descripción} \\
      \hline
      \makecell{Usuario} &  Persona que hará el uso del conjunto de funcionalidades del aplicativo final  \\
      \hline
      \makecell{Nonograma} &  \makecell{Rompecabezas de MxN dimensiones}  \\
      \hline
      \makecell{Nivel} &  Nonograma a crear o resolver por el Usuario\\
      \hline
      \makecell{Progreso} &  \makecell{Datos relacionados con la persistencia de la resolución de un nivel} \\
      \hline
    \end{tabular}
    \label{fig:glo2}
\end{table}

Por otra parte, los términos técnicos que componen el sistema son los que siguen:

\begin{table}[H]
    \caption{Glosario de términos técnicos del aplicativo}
      \begin{tabular}{ | c | c |}
        \hline
        \thead{Término} & \thead{Descripción} \\
        \hline
        \makecell{Nombre \\ de usuario} &  \makecell{Nombre que adoptará el Usuario de forma opcional para su\\identificación en el aplicativo.}  \\
        \hline
        \makecell{Email} &  \makecell{Cuenta de correo que hará uso el usuario para acceder a los servicios \textit{en nube}}  \\
        \hline
        \makecell{Fecha de \\ publicación} &  Fecha en la que el usuario crea y publica un nivel  \\
        \hline
        \makecell{Figura} &  Nombre identificativo de un nonograma \\
        \hline
        \makecell{Vidas} &  Número de intentos en la resolución de un nivel \\
        \hline
        
      \end{tabular}
      \label{fig:table2}
  \end{table}

  \subsection{Modelo de Dominio}
  Una vez introducida la términología propia del sistema, para un mayor entendimiento del contexto del mismo, se representa un  
  diagrama de clases de acuerdo a las reglas clásicas UML, como se pude visualizar en la \autoref{fig:dom1}.

  \begin{figure}[H]
    \centering
  \begin{tikzpicture}
    \begin{umlpackage}[fill=black!10]{Sistema}
      \umlclass[x=1,y=1]{Usuario}{uuid : string \\ nombreUsuario : string? \\ email : string? \\ numCompletados : int}{}
      \umlclass[x=2,y=-3.8]{Nivel}{uuid : string \\ fechaPub : datetime? \\ figura : string \\ vidas : int}{}
      \umlclass[x=9,y=-3.8]{Nonograma}{  width : int \\ height : int \\ celdasCorrectas : List<int>}{}
      \umlclass[x=8,y=1]{<<data>> Progreso}{id : int \\ completado : int \\ celdasMarcadas : string? \\ celdasDescartadas : string? \\ vidas : int}{}
    \end{umlpackage}
      
    \umlassoc[pos=0.6,arg=crea,mult1=1,mult2=0*, anchors=-130 and 130] {Usuario}{Nivel}
    \umlassoc[pos=0.6,arg=resuelve,mult1=1,mult2=0*, anchors=-60 and 55] {Usuario}{Nivel}
    \umlassoc[pos=0.7,arg=contiene,mult1=1,mult2=1] {Nivel}{Nonograma}
    \umlassoc[arg=tiene,mult1=1,mult2=0..1] {Nivel}{<<data>> Progreso}
    % \umlassoc[geometry=-|-, arg1=tata, mult1=*, pos1=0.3, arg2=toto, mult2=1, pos2=2.9, align2=left]{C}{B}
    % \umlunicompo[geometry=-|, arg=titi, mult=*, pos=1.7, stereo=vector]{D}{C}
    % \umlaggreg[arg=tutu, mult=1, pos=0.8, angle1=30, angle2=60, loopsize=2cm]{D}{D}
    % \umlinherit[geometry=-|]{D}{B}
    \end{tikzpicture}
    \caption{Diagrama del modelo de dominio}
   \label{fig:dom1}
  \end{figure}

  A continuación, se comenta la función de cada de una de las clases conceptuales y se enumeran en la \autoref{fig:table3} y \autoref{fig:table6} 
  los atributos que las componen. \textit{(La anotación ? al lado del tipo refleja su posibilidad de
  nulidad en el sistema.)}
  

  \textbf{$\bullet$ Clase Usuario}: Conjunto de datos del usuario relacionados con el sistema.

  \begin{table}[H]
    \centering
    \caption{Atributos de la clase Usuario}
      \begin{tabular}{ | c | c |}
        \hline
        \thead{Atributos de Usuario} & \thead{Descripción} \\
        \hline
        \makecell{uuid\\\textit{\textit{string}}} & \makecell{Cadena de caracteres único que identifica al usuario a \\nivel interno}\\
        \hline
        \makecell{nombre de Usuario\\\textit{string?}} & Pseudónimo único a elegir por el usuario en el sistema \\
        \hline
        \makecell{email\\\textit{string?}} &  Cuenta de correo de registro única para el usuario\\
        \hline
        \makecell{numCompletados\\\textit{int}} &  \makecell{Número de niveles completados por el usuario} \\
        \hline
      \end{tabular}
      \label{fig:table3}
  \end{table}

  \textbf{$\bullet$ Clase Nivel}: Muestra las características de un determinado nivel a resolver.

  \begin{table}[H]
    \centering
    \caption{Atributos de la clase Nivel}
      \begin{tabular}{ | c | c |}
        \hline
        \thead{Atributos de Nivel} & \thead{Descripción} \\
        \hline
        \makecell{uuid\\\textit{\textit{string}}} & \makecell{Cadena de caracteres único que identifica al nivel}\\
        \hline
        \makecell{fecha de Publicación\\\textit{dateTime?}} & Fecha de creación del nonograma en caso de llegar a publicarse \\
        \hline
        \makecell{figura\\\textit{string}} &  Nombre del nivel mostrado una vez resuelto\\
        \hline
        \makecell{vidas\\\textit{int}} &  \makecell{Número de intentos que dispone un usuario para la resolución de \\un nivel} \\
        \hline
      \end{tabular}
      \label{fig:table4}
  \end{table}

  \textbf{$\bullet$ Clase Progreso}: Clase de persistencia que contiene los datos durante la resolución de un nivel.

  \begin{table}[H]
    \centering
    \caption{Atributos de la clase Progreso}
      \begin{tabular}{ | c | c |}
        \hline
        \thead{Atributos de Progreso} & \thead{Descripción} \\
        \hline
        \makecell{id\\\textit{int}} & \makecell{Número de identificación del progreso de un determinado nivel}\\
        \hline
        \makecell{completado\\\textit{int}} & \makecell{Indica con un 1 si el nivel está resuelto y 0 si no lo está\\
        (simulando un dato de tipo booleano)}\\
        \hline
        \makecell{celdasMarcadas\\\textit{string}} &  \makecell{Números de las celdas pintadas separados por delimitadores\\
        (simulando un dato de tipo lista de enteros)}\\
        \hline
        \makecell{celdasDescartadas\\\textit{string}} &  \makecell{Números de las celdas descartadas separados por delimitadores\\
        (simulando un dato de tipo lista de enteros)}\\
        \hline
        \makecell{vidas\\\textit{int}} &  \makecell{Números de vidas que dispone el usuario en ese momento}\\
        \hline
      \end{tabular}
      \label{fig:table5}
  \end{table}
  \hfill 

  \textbf{$\bullet$ Clase Nonograma}: Contiene los datos del nonograma de un determinado nivel.

  \begin{table}[H]
    \centering
    \caption{Atributos de la clase Nonograma}
      \begin{tabular}{ | c | c |}
        \hline
        \thead{Atributos de Nonograma} & \thead{Descripción} \\
        \hline
        \makecell{width\\\textit{int}} & \makecell{Número de filas de celdas que compone el Nonograma}\\
        \hline
        \makecell{height\\\textit{int}} & Número de columnas de celdas que compone el Nonograma \\
        \hline
        \makecell{celdasCorrectas\\\textit{List<int>}} & \makecell{Lista de los números de celdas que componen \\
        la solución del Nonograma} \\
        \hline
      \end{tabular}
      \label{fig:table6}
  \end{table}

  \subsection{Límites del Sistema}
  Para representar los límites de la aplicación, se emplea un diagrama de contexto en el que se muestra la jerarquía de los actores 
  que van a interactúan con el aplicativo.

\tikzumlset{fill usecase=white}
\begin{figure}[H]
  \centering
\begin{tikzpicture}
    \begin{umlsystem}[x=5,y=-3,fill=black!10] {Aplicativo} % empty title
        % \umlusecase[name=a,width=1.5cm] {Use case}
        % \umlusecase[name=b,x=6,width=2.5cm] {Use case b}
        % \umlusecase[name=c,x=6,y=-3,width=2.5cm] {Use case c}
        % \umlusecase[name=d,y=-3,width=2.5cm] {Use case d}
    \end{umlsystem}


    \umlactor[y=-1] {Usuario}
    \umlactor[y=-4] {Usuario Registrado}

    \umlassoc{Usuario}{Aplicativo}
    \umlinherit{Usuario Registrado}{Usuario}
    
    % \umlextend{a}{b}
    % \umlinclude{c}{d}

    % manual versions of the above
    % \draw [tikzuml dependency style] (a) -- node[above] {$\ll \text{extend} \gg$} (b);
    % \draw [tikzuml dependency style] (d) -- node[above] {$\ll \text{include} \gg$} (c);

    % bent association
  \end{tikzpicture}
  \caption{Diagrama de contexto del aplicativo}
   \label{fig:dom2}
\end{figure}

Como se puede apreciar en el diagrama de la \autoref{fig:dom2}, únicamente van a interactuar dos actores con el sistema: i) el usuario sin registro,
que emplea únicamente las funciones locales del sistema y ii) el usuario registrado, que podrá emplear tanto las funciones locales como
\textit{en nube} del aplicativo.

\subsection{Restricciones del Sistema}

En este apartado se muestran algunas de las restricciones que pueden impedir algunas de las funcionalidades que ofrece el sistema
al usuario, vistos en la \autoref{fig:table7}.

\begin{table}[H]
  \centering
  \caption{Restricciones del sistema}
    \begin{tabular}{ | c | c |}
      \hline
      \thead{Restricción} & \thead{Descripción} \\
      \hline
      \makecell{Memoria del\\ dispositivo} & \makecell{Posibilidad de que el dispositivo no tenga suficiente memoria
      \\disponible para guardar el progreso de la resolución de los niveles.\\\textit{Probabilidad prácticamente despreciable} }\\
      \hline
      \makecell{Conexión a\\internet} & \makecell{Las funcionalidades \textit{en nube} requieren de conexión a internet\\ para poder 
      llegar a usarse.} \\
      \hline
      \makecell{Cuenta\\de Google} &  \makecell{El usuario pueder carecer de una cuenta de correo de Google\\necesaria para hacer uso de las funcionalidades 
      \textit{en nube}.} \\
      \hline
    \end{tabular}
    \label{fig:table7}
\end{table}

\subsection{Características del Sistema}
Una vez vistas las limitaciones y restricciones del sistema, se pueden analizar las características finales 
a incorporar en la aplicación, mediante \textit{casos de uso}.

\subsubsection{Casos de uso principales}

Estos, son los que se encontrarán en la \textit{Pantalla Principal} y \textit{Pantalla de Seleción de Nonogramas} 
del aplicativo.

\begin{figure}[H]
  \centering
\begin{tikzpicture}
  \begin{umlsystem}[x=5, fill=black!10]{Aplicativo} 
    \umlusecase[name = usecase1, width=2cm]{Jugar nivel \textbf{R01}}
    \umlusecase[name = usecase2, x=6,width=2.4cm]{Seleccionar nivel clásico \textbf{R05}}
    \umlusecase[name = usecase3, x=4, y=-2, width=2cm]{Acceder Tutorial \textbf{R02}} 
    \umlusecase[name = usecase4, y=-5]{Acceder a En Línea \textbf{R03}}
    \umlusecase[name = usecase5, x=6, y=-4, width=2cm]{Publicar nonograma \textbf{R06}}
    \umlusecase[name = usecase6, x=6, y=-5.5, width=3cm]{Resolver nivel online \textbf{R07}}
    \umlusecase[name = usecase7, y=-3]{Acceder a Ajustes \textbf{R04}}
    \end{umlsystem}
    \umlactor[y=-1]{Usuario}
    \umlactor[y=-4]{Usuario Registrado} 
    \umlinherit{Usuario Registrado}{Usuario}

    \umlinclude[]{usecase1}{usecase2}
    \umlassoc[]{Usuario}{usecase1}
    \umlassoc[]{Usuario Registrado}{usecase4}
    \umlinclude[]{usecase4}{usecase5}
    \umlinclude[]{usecase4}{usecase6}
    \umlassoc[]{Usuario}{usecase3}
    \umlassoc[]{Usuario}{usecase7}

\end{tikzpicture}
\caption{Diagrama de casos de usos principales}
\label{fig:caso1}
\end{figure}

\subsubsection{Casos de uso de resolución de nivel}

Estarán disponibles en la pantalla de resolución de nivel.

\begin{figure}[H]
  \centering
\begin{tikzpicture}
  \begin{umlsystem}[x=5, fill=black!10]{Aplicativo} 
    \umlusecase[name = usecase4, y=-4]{Jugar nonograma \textbf{R08}}
    \umlusecase[name = usecase5, x=7, y=-3, width=2cm]{Superar nivel \textbf{R09}}
    \umlusecase[name = usecase6, x=4, y=-6]{Perder nivel \textbf{R10}}
    \umlusecase[name = usecase7, y=-6, width=1.5cm]{Cargar progreso \textbf{R11}}
    \umlusecase[name = usecase8, y=-2, width=1.5cm]{Salir del nivel \textbf{R12}}
    \umlusecase[name = usecase9, y=-2, x=4, width=1.5cm]{Guardar progreso \textbf{R13}}
    \end{umlsystem}
    \umlactor[y=-4]{Usuario} 

    \umlassoc[]{Usuario Registrado}{usecase4}
    \umlVHextend[]{usecase5}{usecase4}
    \umlVHextend[]{usecase6}{usecase4}
    \umlinclude[]{usecase4}{usecase7}
    \umlinclude[]{usecase4}{usecase8}
    \umlVHextend[]{usecase9}{usecase8}

\end{tikzpicture}
\caption{Diagrama de casos de usos en pantalla de resolución de nivel}
\label{fig:caso2}
\end{figure}

\subsubsection{Casos de uso de ajustes}

Casos de uso visibles en la \textit{Pantalla de Ajustes} del sistema.

\begin{figure}[H]
  \centering
\begin{tikzpicture}
  \begin{umlsystem}[x=5, fill=black!10]{Aplicativo} 
    \umlusecase[name = usecase1, x=-1]{Cambiar tema \textbf{R14}}
    \umlusecase[name = usecase3, x=4, y=-0.5, width=2cm]{Seleccionar vidas \textbf{R15}}
    \umlusecase[name = usecase9, x=1, y=-1.4, width=2cm]{Iniciar sesión \textbf{R16}} 
    \umlusecase[name = usecase4, x=1, y=-6]{Cerrar sesión \textbf{R20}}
    \umlusecase[name = usecase5, x=3, y=-5]{Sincronizar datos \textbf{R21}}
    \umlusecase[name = usecase6, x=4, y=-3, width= 2.5cm]{Seleccionar auto-guardado \textbf{R17}}
    \umlusecase[name = usecase7, y=-3]{Borrar datos \textbf{R18}}
    \umlusecase[name = usecase8, x=-1, y=-3.8]{Seleccionar idioma \textbf{R19}}
    \end{umlsystem}
    \umlactor[y=-1,x=-1]{Usuario}
    \umlactor[y=-4, x=-1]{Usuario Registrado} 
    \umlinherit{Usuario Registrado}{Usuario}

    \umlassoc[]{Usuario}{usecase1}
    \umlassoc[]{Usuario Registrado}{usecase4}
    \umlassoc[]{Usuario Registrado}{usecase5}
    \umlassoc[]{Usuario}{usecase3}
    \umlassoc[]{Usuario}{usecase6}
    \umlassoc[]{Usuario}{usecase7}
    \umlassoc[]{Usuario}{usecase8}
    \umlassoc[]{Usuario}{usecase9}

\end{tikzpicture}
\caption{Diagrama de casos de ajustes}
\label{fig:caso3}
\end{figure}

\textit{Cada uno de los requisitos funcionales (RF) han sido numerados con un identificador \textbf{RXX}, y se harán referencia a ellos
en su especificación.}

\subsection{Requisitos funcionales}

Para evitar posibles inconsistencias a nivel de producto, se especificarán cada uno de los requisitos funcionales previamente 
numerados, enumerando sus datos de entrada y salida, precondiciones, postcondiciones, flujos y criterios de aceptación.

\medspace

\newcommand\addrow[2]{#1 &#2\\ }

\newcommand\addheading[2]{#1 &#2\\ \hline}
\newcommand\tabularhead{\begin{tabular}{lp{9cm}}
\hline
}

\newcommand\addmulrow[2]{ \begin{minipage}[t][][t]{4cm}#1\end{minipage}% 
   &\begin{minipage}[t][][t]{8.5cm}
    \begin{enumerate} #2   \end{enumerate}
    \end{minipage}\\ }
    

\newenvironment{usecase}{\tabularhead}
{\hline\end{tabular}}

\begin{table}[H]
\begin{usecase}
  \addheading{Actor: \textbf{Usuario}}{\textbf{R1 - Jugar nivel}}
  \addrow{Descripción}{Acceder al listado de niveles clásicos disponibles en el sistema}
  \addrow{Precondiciones}{Ninguna}
  \addrow{Entradas}{Ninguna}
  \addrow{Salidas}{Listado de sensores predeterminados clásicos}
  \addrow{Postcondiciones}{Se mostrará el listado de niveles}
  \addmulrow{Flujo principal}{\item Presionar la sección Jugar
                                   }
                                  \addrow{}{}
\end{usecase}
\caption{Tabla de requisito funcional \textit{Jugar nivel}}
\label{table:req1}
\end{table}

\begin{table}[H]
  \begin{usecase}
    \addheading{Actor: \textbf{Usuario}}{\textbf{R2 - Visualizar Tutorial}}
    \addrow{Descripción}{Muestra el tutorial para familiarizar al usuario de las reglas de
    la resolución de nonogramas}
    \addrow{Precondiciones}{Ninguna}
    \addrow{Entradas}{Ninguna}
    \addrow{Salidas}{Listado de reglas y consejos}
    \addrow{Postcondiciones}{Se mostrarán los consejos y reglas }
    \addmulrow{Flujo principal}{\item Presionar la sección Tutorial
    \item Deslizar el conjunto de consejos y reglas
                                     }
                                    \addrow{}{}
  \end{usecase}
  \caption{Tabla de requisito funcional \textit{Visualizar Tutorial}}
  \label{table:req2}
  \end{table}

  \begin{table}[H]
    \begin{usecase}
      \addheading{Actor: \textbf{Usuario Registrado}}{\textbf{R3 - Acceder sección En Línea}}
      \addrow{Descripción}{Muestra el conjunto de funciones en red}
      \addrow{Precondiciones}{El usuario debe estar registrado }
      \addrow{Entradas}{Número identificativo del usuario registrado}
      \addrow{Salidas}{Listado de funcionalidades en línea}
      \addrow{Postcondiciones}{Se mostrará el listado de funcionalidades }
      \addmulrow{Flujo principal}{\item Presionar la sección En Línea
                                       }
                                      \addrow{}{}
    \end{usecase}
    \caption{Tabla de requisito funcional \textit{Acceder sección En Línea}}
    \label{table:req3}
    \end{table}

    \begin{table}[H]
      \begin{usecase}
        \addheading{Actor: \textbf{Usuario}}{\textbf{R4 - Acceder sección Ajustes}}
        \addrow{Descripción}{Muestra el conjunto de configuraciones propias del sistema}
        \addrow{Precondiciones}{Ninguna}
        \addrow{Entradas}{Ninguna}
        \addrow{Salidas}{Listado de configuraciones}
        \addrow{Postcondiciones}{Se mostrará el listado de configuraciones }
        \addmulrow{Flujo principal}{\item Presionar la sección Ajustes
                                         }
                                        \addrow{}{}
      \end{usecase}
      \caption{Tabla de requisito funcional \textit{Acceder sección Ajustes}}
      \label{table:req4}
      \end{table}
  
  \begin{table}[H]
      \begin{usecase}
        \addheading{Actor: \textbf{Usuario}}{\textbf{R5 - Seleccionar nivel clásico}}
        \addrow{Descripción}{Permite seleccionar un nivel predeterminado del aplicativo}
        \addrow{Precondiciones}{Ninguna}
        \addrow{Entradas}{Ninguna}
        \addrow{Salidas}{Listado de niveles predeterminados}
        \addrow{Postcondiciones}{Se mostrará el listado de niveles }
        \addmulrow{Flujo principal}{
        \item Seleccionar un nivel del listado
                                         }
                                        \addrow{}{}
      \end{usecase}
      \caption{Tabla de requisito funcional \textit{Seleccionar nivel clásico}}
      \label{table:req5}
      \end{table}

  \begin{table}[H]
      \begin{usecase}
        \addheading{Actor: \textbf{Usuario Registrado}}{\textbf{R6 - Publicar nonograma}}
        \addrow{Descripción}{Permite crear un nonograma con características específicas}
        \addrow{Precondiciones}{El usuario debe estar registrado}
        \addrow{Entradas}{Nombre del nonograma, dimensiones, celdas correctas y nombre del autor}
        \addrow{Salidas}{Nonograma creado y disponible en el listado de nonogramas en línea}
        \addrow{Postcondiciones}{Se mostrará el listado de niveles }
        \addmulrow{Flujo principal}{
        \item Seleccionar la sección Crear nonograma
        \item Establecer nombre y dimensiones del nonograma y autor
        \item Seleccionar cada una de las celdas indicando que son correctas
                                         }
                                        \addrow{}{}
      \end{usecase}
      \caption{Tabla de requisito funcional \textit{Publicar nonograma}}
      \label{table:req6}
      \end{table}

  \begin{table}[H]
      \begin{usecase}
        \addheading{Actor: \textbf{Usuario Registrado}}{\textbf{R7 - Resolver nivel online}}
        \addrow{Descripción}{Permite descargar y resolver un nivel creado por el mismo
        usuario o por la comunidad}
        \addrow{Precondiciones}{El usuario debe estar registrado}
        \addrow{Entradas}{Nonograma seleccionado}
        \addrow{Salidas}{Nonograma descargado}
        \addrow{Postcondiciones}{Se mostrará el nivel online a resolver}
        \addmulrow{Flujo principal}{
        \item Seleccionar un nivel del listado de niveles online
                                         }
                                        \addrow{}{}
      \end{usecase}
      \caption{Tabla de requisito funcional \textit{Resolver nivel online}}
      \label{table:req7}
      \end{table}

\begin{table}[H]
      \begin{usecase}
        \addheading{Actor: \textbf{Usuario}}{\textbf{R8 - Jugar nonograma}}
        \addrow{Descripción}{Muestra la pantalla y permite la resolución del nonograma de forma
        interactiva}
        \addrow{Precondiciones}{Se debe haber seleccionado un nivel}
        \addrow{Entradas}{Nivel seleccionado}
        \addrow{Salidas}{Nonograma a resolver}
        \addrow{Postcondiciones}{Se mostrará el nonograma a resolver}
        \addmulrow{Flujo principal}{
        \item Pulsar en las celdas para descubrir las celdas correctas y resolver el nonograma
        \item Pulsar dos veces o mantener pulsado en las celdas para indicar
        la celda como no correcta
                                         }
                                        \addrow{}{}
      \end{usecase}
      \caption{Tabla de requisito funcional \textit{Jugar nonograma}}
      \label{table:req8}
      \end{table}

\begin{table}[H]
      \begin{usecase}
        \addheading{Actor: \textbf{Usuario}}{\textbf{R9 - Superar nivel}}
        \addrow{Descripción}{Muestra una pantalla de felicitación una vez resuelto el nonograma}
        \addrow{Precondiciones}{Todas las celdas correctas del nonograma se deben haber seleccionado
        y se debe disponer de vidas}
        \addrow{Entradas}{Nivel seleccionado, celdas correctas y vidas}
        \addrow{Salidas}{Nonograma descubierta}
        \addrow{Postcondiciones}{Se mostrará una pantalla de felicitación y la información del nonograma
        resuelto}
        \addmulrow{Flujo principal}{
        \item Pulsar sobre cada una de las celdas correctas del nonograma superando el nivel
                                         }
                                        \addrow{}{}
      \end{usecase}
      \caption{Tabla de requisito funcional \textit{Superar nivel}}
      \label{table:req9}
      \end{table}

\begin{table}[H]
      \begin{usecase}
        \addheading{Actor: \textbf{Usuario}}{\textbf{R10 - Perder nivel}}
        \addrow{Descripción}{Muestra una pantalla indicando que has fallado en la resolución del nivel
        y permite intentarlo de nuevo}
        \addrow{Precondiciones}{El número de vidas especificado en ajustes debe ser finito y
        durante el nivel el usuario debe haberse quedado sin vidas}
        \addrow{Entradas}{Número de vidas}
        \addrow{Salidas}{Opción intentar de nuevo}
        \addrow{Postcondiciones}{Se mostrará una pantalla de repetir nivel}
        \addmulrow{Flujo principal}{
        \item Pulsar sobre celdas incorrectas hasta quedarse sin vidas
                                         }
                                        \addrow{}{}
      \end{usecase}
      \caption{Tabla de requisito funcional \textit{Perder nivel}}
      \label{table:req10}
      \end{table}

\begin{table}[H]
      \begin{usecase}
        \addheading{Actor: \textbf{Usuario}}{\textbf{R11 - Cargar progreso}}
        \addrow{Descripción}{Muestra un pop-up permitiendo al usuario poder cargar el progreso
        de la resolución del nivel}
        \addrow{Precondiciones}{El nivel seleccionado para su resolución debe de haberse jugado
        anteriormente y tener un progreso}
        \addrow{Entradas}{Nivel específico y progreso del mismo}
        \addrow{Salidas}{Último progreso de la resolución del nivel seleccionado}
        \addrow{Postcondiciones}{Se mostrará la pantalla de resolución del nivel con
        el último progreso guardado}
        \addmulrow{Flujo principal}{
        \item Pulsar sobre la opción cargar progreso en un nivel concreto
                                         }
                                        \addrow{}{}
      \end{usecase}
      \caption{Tabla de requisito funcional \textit{Cargar progreso}}
      \label{table:req11}
      \end{table}

\begin{table}[H]
      \begin{usecase}
        \addheading{Actor: \textbf{Usuario}}{\textbf{R12 - Salir del nivel}}
        \addrow{Descripción}{Permite al usuario salir del nivel}
        \addrow{Precondiciones}{El usuario debe de encontrarse en la pantalla de 
        resolución de un nivel}
        \addrow{Entradas}{Ninguna}
        \addrow{Salidas}{Ninguna}
        \addrow{Postcondiciones}{Se mostrará el listado de niveles}
        \addmulrow{Flujo principal}{
        \item Pulsar sobre la opción salir del nivel
                                         }
                                        \addrow{}{}
      \end{usecase}
      \caption{Tabla de requisito funcional \textit{Salir del nivel}}
      \label{table:req12}
      \end{table}

\begin{table}[H]
      \begin{usecase}
        \addheading{Actor: \textbf{Usuario}}{\textbf{R13 - Guardar progreso}}
        \addrow{Descripción}{Permite al usuario guardar su progreso durante la resolución de un nivel}
        \addrow{Precondiciones}{El usuario debe de encontrarse en la pantalla de 
        resolución de un nivel y debe de haberse habilitado la opción Autoguardado en ajustes}
        \addrow{Entradas}{Ninguna}
        \addrow{Salidas}{Ninguna}
        \addrow{Postcondiciones}{Se guardará el progreso del nivel}
        \addmulrow{Flujo principal}{
        \item Salir del nivel teniendo habilitada la opción de Autoguardado habilitada
                                         }
                                        \addrow{}{}
      \end{usecase}
      \caption{Tabla de requisito funcional \textit{Guardar progreso}}
      \label{table:req13}
      \end{table}

\begin{table}[H]
      \begin{usecase}
        \addheading{Actor: \textbf{Usuario}}{\textbf{R14 - Cambiar tema}}
        \addrow{Descripción}{Concede al usuario la opción de elegir un tema visual dentro de los disponibles 
        del sistema}
        \addrow{Precondiciones}{Ninguna}
        \addrow{Entradas}{Tema a elegir}
        \addrow{Salidas}{Ninguna}
        \addrow{Postcondiciones}{Se actualizará el tema global del sistema}
        \addmulrow{Flujo principal}{
        \item Seleccionar el tema a elegir mediante el botón Cambiar, eligiendo de forma dinámica la
        combinación de colores
                                         }
                                        \addrow{}{}
      \end{usecase}
      \caption{Tabla de requisito funcional \textit{Cambiar tema}}
      \label{table:req14}
      \end{table}

\begin{table}[H]
      \begin{usecase}
        \addheading{Actor: \textbf{Usuario}}{\textbf{R15 - Seleccionar vidas}}
        \addrow{Descripción}{Concede al usuario la opción de elegir el número de vidas disponibles durante la
        resolución de un nivel}
        \addrow{Precondiciones}{Ninguna}
        \addrow{Entradas}{Número de vidas a elegir}
        \addrow{Salidas}{Ninguna}
        \addrow{Postcondiciones}{Se actualizará el número de vidas por defecto en la resolución de un nivel}
        \addmulrow{Flujo principal}{
        \item Seleccionar el número de vidas mediante el botón cambiar (de 1 a 5 vidas) o infinitas.
                                         }
                                        \addrow{}{}
      \end{usecase}
      \caption{Tabla de requisito funcional \textit{Seleccionar vidas}}
      \label{table:req15}
      \end{table}

\begin{table}[H]
      \begin{usecase}
        \addheading{Actor: \textbf{Usuario}}{\textbf{R16 - Iniciar sesión}}
        \addrow{Descripción}{Permite al usuario registrarse en el caso de no estar registrado mediante una cuenta
        de Google e identificarse dentro del aplicativo para acceder a las funciones online}
        \addrow{Precondiciones}{Ninguna}
        \addrow{Entradas}{Credenciales de Google}
        \addrow{Salidas}{Ninguna}
        \addrow{Postcondiciones}{Se actualizará la cuenta que ha iniciado sesión el usuario a nivel de sistema}
        \addmulrow{Flujo principal}{
        \item Seleccionar la opción de Ingresar en Cuenta de usuario
        \item Presionar Iniciar sesión con Google
                                         }
                                        \addrow{}{}
      \end{usecase}
      \caption{Tabla de requisito funcional \textit{Iniciar sesión}}
      \label{table:req16}
      \end{table}

\begin{table}[H]
      \begin{usecase}
        \addheading{Actor: \textbf{Usuario}}{\textbf{R17 - Seleccionar autoguardado}}
        \addrow{Descripción}{Permite al usuario elegir habilitar o deshabilitar la opción de autoguardado en la
        resolución de niveles}
        \addrow{Precondiciones}{Ninguna}
        \addrow{Entradas}{Ninguna}
        \addrow{Salidas}{Ninguna}
        \addrow{Postcondiciones}{Se actualizará la opción de autoguardado en las preferencias del sistema}
        \addmulrow{Flujo principal}{
        \item Habilitar o deshabilitar la opción de autoguardado
                                         }
                                        \addrow{}{}
      \end{usecase}
      \caption{Tabla de requisito funcional \textit{Seleccionar autoguardado}}
      \label{table:req17}
      \end{table}

\begin{table}[H]
  \begin{usecase}
    \addheading{Actor: \textbf{Usuario}}{\textbf{R18 - Borrar datos}}
    \addrow{Descripción}{Borra el progreso de todos los niveles jugados}
    \addrow{Precondiciones}{Ninguna}
    \addrow{Entradas}{Ninguna}
    \addrow{Salidas}{Ninguna}
    \addrow{Postcondiciones}{Se debe de perder el progreso de todos los niveles jugados}
    \addmulrow{Flujo principal}{
    \item Presionar la opción borrar en la sección borrar progreso
                                      }
                                    \addrow{}{}
  \end{usecase}
  \caption{Tabla de requisito funcional \textit{Borrar datos}}
  \label{table:req18}
  \end{table}

\begin{table}[H]
  \begin{usecase}
    \addheading{Actor: \textbf{Usuario}}{\textbf{R19 - Seleccionar idioma}}
    \addrow{Descripción}{Cambia el idioma global del aplicativo}
    \addrow{Precondiciones}{Ninguna}
    \addrow{Entradas}{Ninguna}
    \addrow{Salidas}{Ninguna}
    \addrow{Postcondiciones}{Se debe de perder el progreso de todos los niveles jugados}
    \addmulrow{Flujo principal}{
    \item Presionar la opción cambiar, seleccionando el idioma deseado
                                      }
                                    \addrow{}{}
  \end{usecase}
  \caption{Tabla de requisito funcional \textit{Seleccionar idioma}}
  \label{table:req19}
  \end{table}

\begin{table}[H]
  \begin{usecase}
    \addheading{Actor: \textbf{Usuario Registrado}}{\textbf{R20 - Cerrar sesión}}
    \addrow{Descripción}{Cierra sesión del usuario, cambiando los datos del jugador a nivel global del
    aplicativo }
    \addrow{Precondiciones}{El usuario debe estar registrado}
    \addrow{Entradas}{Ninguna}
    \addrow{Salidas}{Ninguna}
    \addrow{Postcondiciones}{Mostrar al usuario que su sesión ha sido finalizada}
    \addmulrow{Flujo principal}{
    \item Presionar en Ingresar en el apartado Cuenta de usuario
    \item Presionar en Finalizar sesión con Google
                                      }
                                    \addrow{}{}
  \end{usecase}
  \caption{Tabla de requisito funcional \textit{Cerrar sesión}}
  \label{table:req20}
  \end{table}

\begin{table}[H]
  \begin{usecase}
    \addheading{Actor: \textbf{Usuario}}{\textbf{R21 - Sincronizar datos}}
    \addrow{Descripción}{Guarda el progreso de los niveles clásicos mediante el servicio en nube}
    \addrow{Precondiciones}{El usuario debe estar registrado}
    \addrow{Entradas}{Ninguna}
    \addrow{Salidas}{Ninguna}
    \addrow{Postcondiciones}{Mostrar un popup indicando que los datos han sido sincronizados}
    \addmulrow{Flujo principal}{
    \item Presionar la Sincronizar
                                      }
                                    \addrow{}{}
  \end{usecase}
  \caption{Tabla de requisito funcional \textit{Sincronizar datos}}
  \label{table:req21}
  \end{table}
