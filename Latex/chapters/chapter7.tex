\chapter{Implantación y mantenimiento}
\textit{En un ciclo de vida de un entorno de Integración Continua (CI),
tras una inserción de código en un repositorio, posteriormente revisado por
un tercero, un sistema funcional coge el cambio introducido, comprueba
el código y efectúa una serie de comandos que el cambio es positivo
y no resulta perjudicial para la solución.~\cite{6802994}}

\section{Integración Continua}
Ya que se ha empleado tanto metodología \textit{PXP} como \textit{TDD} para este proyecto,
resultó idóneo el incorporar un aspecto tan positivo como es
la \textit{Integración Continua}.

Como se había comentado anteriormente en la \autoref{sec:git},
la solución se aloja en un repositorio de la plataforma \textit{GitLab}.
Como la mayoría de plataformas de servicios web de control de versiones,
presenta una serie de herramientas propias de prácticas \textit{DevOps} que promueven
el contínuo mantenimiento y seguridad del código, esta propiedad es
posible mediante el servicio \textit{Auto DevOps}.

A través de la opción de \textit{Quality Gate} de \textit{GitLab},
se establecieron ciertas condiciones que debe cumplir cualquier
cambio que se desee introducir a la rama \textit{develop}. Estas comprobaciones
están definidas en un archivo \textit{YAML} en el archivo raíz del
proyecto. De tal forma que cada vez que se realiza un \textit{merge} de una
determinada \textit{feature} o directamente un \textit{commit} hacia la rama
\textit{develop} se ejecutan la totalidad de tests desarrollados. En este
momento, se desarrollan dos vertientes:

\begin{itemize}
    \item[$\bullet$] Si las condiciones fallan no se admite la inserción de los
    cambios a la rama protegida a \textit{develop}, devolviendo un mensaje de error con el test o tests
    fallidos.
    \item[$\bullet$] Si las condiciones se cumplen, se inserta el cambio o los cambios deseados en la rama \textit{develop} con éxito.
\end{itemize}

\section{Actualización del proyecto}
Por último, se quería destacar la adaptación del proyecto de cara a las últimas
versiones de \textit{stable} en \textit{Flutter}, más concretamente la versión \textit{Flutter 2.0.0}.

Esta actualización se estableció el 3 de marzo de 2021 e incorporaba muchas novedades y
cambios, como: soporte para entorno \textit{web}, soporte enfocadas a plataformas:
\textit{macOS}, \textit{Windows} y \textit{Linux}, manejo de anuncios \textit{Add-To-App}
y la incorporación de \textit{null-safety} en \textit{Dart}.

Es esta última característica de la versión la que se quiso incorporar en el proyecto,
esta característica permitiría evitar \textit{aserciones} continuas en el código,
definiendo clases con atributos que pueden llegar a adquirir valores nulos, asegurando
que no ocurren excepciones durante su ejecución.

Para la inclusión de esta última actualización en el proyecto se realizó mediante
una herramienta de migración desarrollada por \textit{Flutter}.
\textit El equipo de {Dart} sugirió una serie de pasos para la inclusión de la característica \textit{null-safety}
en el proyecto~\cite{dartmigrate}.

\begin{enumerate}
    \item Esperar a que los paquetes incluidos en el proyecto adopten la última
    actualización con la etiqueta \textit{null-safety}.
    \item Ejecutar el cliente de migración  
    \item De forma estática analiza el código de tu proyecto.
    \item Prueba que los cambios realizados funcionan.
    \item Si el proyecto está publicado en \textit{pub.dev}, publica el proyecto con
    la etiqueta \textit{null-safety} como una versión \textit{prerelease}.
\end{enumerate}

El resultado de esta migración promovió que todos los paquetes incluidos en el
proyecto estén actualizados con la última actualización \textit{null-safety},
un código más limpio y libre de código innecesario, menos excepciones capturadas relacionadas
con valores nulos.

\section{Publicación}
El último paso sería incluir el aplicativo en las principales tiendas de aplicaciones:
\textit{Google Play Store} y \textit{App Store}, para ello se realizó las siguientes pasos
para ambas plataformas.

\begin{enumerate}
    \item Establecer a nivel de proyecto un id de aplicativo para ambas plataformas, junto a sus
    versiones mínimas y máximas de los sistemas operativos que soporta.
    \item Creación de una cuenta con rol de desarrollador,
    \item Generar una versión \textit{release} (un \textit{App Bundle} para Android y 
    un \textit{IPA} para iOS) firmadas con la cuenta de desarrollador deseada.
    \item Configurar  el entorno, opciones y características propias de las plataformas.
\end{enumerate}