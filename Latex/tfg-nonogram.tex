\documentclass[11pt,spanish,listoffigures,listoftables]{tfgetsinf}

\usepackage[utf8]{inputenc} 

%% INFO

\title{Desarrollo de una aplicación móvil multiplataforma \\
para la creación y resolución de nonogramas}
\author{Ignacio Ferrer Sanz}
\tutor{Germán Francisco Vidal Oriola}
\curs{2020-2021}

%% KEYWORDS

\keywords{????, ?????????, ????, ?????????????????}
         {?????, ???, ???????????????}   
         {?????, ????? ?????, ?????????????}

%% BEGIN

\begin{document}

%% SUMMARIES

\begin{abstract}[spanish]
????
\end{abstract}
\begin{abstract}[catalan]
   ????
\end{abstract}
\begin{abstract}[english]
????
\end{abstract}

\mainmatter

\chapter{Introducción}

\textit{Descubrir imágenes hechas píxel de situaciones del día a día, naturaleza, edificios famosos, personas, y más. Esta es la verdadera 
esencia de los nonogramas, también conocidos como hanzies, picross o griddlers.}

\section{Contexto y motivación}

\textit{<<No importa cúan complejo sea de resolver un nonograma, la clave de estos rompecabezas reside en que su resolución pueda efectuarse
por simple lógica.>>} Este era el principal propósito de \textit{James Dalgety} y su equipo de diseñadores, 
responsables de dar a conocer a occidente este conocido pasatiempo nipón impulsado por el arte de la diseñadora \textit{Non Ishida}, más adelante,
responsable de su principal denominación: \textit{"Non"} Ishida y Dia\textit{"gram"}.

No fue hasta mediados del año 1990, que finalmente, se dio a conocer a escala mundial, a través de una publicación del 
periódico británico \textit{The Sunday Telegraph}. Más adelante, el mismo noticiero, adoptó el término bajo el seudónimo de \textit{griddlers},
publicándolos semanalmente.

A lo largo del tiempo, se fue difundiendo el famoso puzzle a nivel exponencial y se puede encontrar en revistas, más periódicos y libros. 
Fue tan notable su crecimiento, que como otros rompecabezas sobrepasó la barrera del formato digital, en forma de sitios web, videojuegos y aplicaciones.

La capacidad creativa que ofrece resulta ilimitada, ya que con tan solo sus celdas dispuestas en forma de matriz \textit{(filas y columnas)},
permite representar todo tipo de figuras, siluetas y formas, como si de un lienzo se tratara. 

Sin embargo, pese a que nos encontramos en plena era digital, son pocos los medios que ofrecen una capacidad de creación,
más allá del simple método tradicional del lápiz y papel.
Por ello, esta característica se hace necesaria para que la cuantía de nonogramas a resolver no ceda, además de otorgar al jugador, no 
solo el rol de \textit{resolutor} sino de \textit{creador}.


\section{Objetivos}

El presente trabajo explora el submundo de las aplicaciones móviles y propone una solución software para: i) permitir 
al usuario de resolverlos de forma interactiva ii) darle la oportunidad de crear sus propios puzzles como lo hizo
\textit{James Dalgety} y su equipo y compartirlos con todos los demás usuarios, promoviendo así una comunidad de entusiastas de este
divertido rompecabezas.

\section{Metodología}

????? ????????????? ????????????? ????????????? ????????????? ????????????? 

\section{Estructura de la memoria} %%%%% Opcional

????? ????????????? ????????????? ????????????? ????????????? ?????????????

%% CHAPTERS

\chapter{??? ???? ??????}

????? ????????????? ????????????? ????????????? ????????????? ?????????????

\section{?? ???? ???? ? ?? ??}

????? ????????????? ????????????? ????????????? ????????????? ?????????????

\chapter{??? ???? ??????}

????? ????????????? ????????????? ????????????? ????????????? ????????????? 

\section{?? ???? ???? ? ?? ??}

????? ????????????? ????????????? ????????????? ????????????? ?????????????

%% CONCLUSION

\chapter{Conclusiones}

????? ????????????? ????????????? ????????????? ????????????? ????????????? 

%% BIBLIOGRAPHY

\begin{thebibliography}{10}

\bibitem{light}
   Jennifer~S. Light.
   \newblock When computers were women.
   \newblock \textit{Technology and Culture}, 40:3:455--483, juliol, 1999.

\bibitem{ifrah}
   Georges Ifrah.
   \newblock \textit{Historia universal de las cifras}.
   \newblock Espasa Calpe, S.A., Madrid, sisena edició, 2008.

\bibitem{WAR}
   Comunicat de premsa del Departament de la Guerra, 
   emés el 16 de febrer de 1946. 
   \newblock Consultat a 
   \url{http://americanhistory.si.edu/comphist/pr1.pdf}.

\end{thebibliography}
\cleardoublepage

%% APPENDIX

\APPENDIX

\chapter{Configuración del sistema}

????? ????????????? ????????????? ????????????? ????????????? ?????????????

\section{Fase de inicialización}

????? ????????????? ????????????? ????????????? ????????????? ?????????????

\section{Identificación de dispositivos}

????? ????????????? ????????????? ????????????? ????????????? ?????????????

%% APPENDIX

\chapter{??? ???????????? ????}

????? ????????????? ????????????? ????????????? ????????????? ????????????? 

%% END

\end{document}
