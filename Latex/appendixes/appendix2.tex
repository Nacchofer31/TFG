\chapter{Fragmentos de código}
\label{cap:anexo2}
\textit{La intención de este anexo es la de mostrar algunos ejemplos de código extraídos del 
resultado de la fase de codificación, con el fin de ofrecer un mayor entendimiento de
los pasos realizados para cada una de las funcionalidades desarrolladas.
}

\section{Código de Pantalla de Resolución de Nivel}

\label{cap:anexo1-1}

Este apartado muestra la codificación de la página de la funcionalidad Resolución de nivel, 
(\textit{play\_game\_page.dart}).


\begin{lstlisting}

    import 'package:flutter/material.dart';
    import 'package:flutter_bloc/flutter_bloc.dart';
    
    import '../../../../common/theme/theme.dart';
    import '../widgets/grid_box_view.dart';
    import '../widgets/info_bar_view.dart';
    import '../bloc/play_game_bloc.dart';
    
    class PlayGamePage extends StatelessWidget {
      @override
      Widget build(BuildContext context) {
        var gridBloc = BlocProvider.of<PlayGameBloc>(context);
        var colorSet = ThemeManager.of(context).colorSet;
        return Scaffold(
            appBar: InformationBar(
              mainColor: colorSet.primaryColor,
              sideColor: colorSet.secondaryColor),
            backgroundColor: colorSet.primaryColor,
            body: SafeArea(
                child: InteractiveViewer(
                    maxScale: 3.6,
                    child: Column(
                        mainAxisAlignment: MainAxisAlignment.spaceEvenly,
                        children: [
                          GridBox(
                            width: gridBloc.width,
                            height: gridBloc.height,
                          ),
                        ]))));
      }
    }
    
    \end{lstlisting}

    \section{Código de Capa de Lógica de Niveles En Línea}

    \label{cap:anexo1-2}
    
    El siguiente apartado contiene la codificación de las entidades, repositorios y casos de uso de la 
    funcionalidad \textit{Niveles en Línea}, (\textit{nonolist\_bloc.dart, nonolist\_state.dart y nonolist\_event.dart}).
    
    \begin{lstlisting}
    
    part of 'nonolist_bloc.dart';

    abstract class NonoListState {}
    
    class LoadNonoListInitialState extends NonoListState {}
    
    class LoadedNonoListState extends NonoListState {
      final List<Nonogram> nonograms;
    
      LoadedNonoListState({required this.nonograms});
    }
    
    class FailedNonoListState extends NonoListState {
      final String errorMessage;
    
      FailedNonoListState({required this.errorMessage});
    }
    
    class LoadingNonoListState extends NonoListState {}
    \end{lstlisting}

    \begin{lstlisting}

    part of 'nonolist_bloc.dart';

    abstract class NonoListEvent {}
    
    class LoadNonoListEvent extends NonoListEvent {
      final String uid;
    
      LoadNonoListEvent({required this.uid});
    }
    
    class RefreshNonoListEvent extends NonoListEvent {}
    \end{lstlisting}

    \begin{lstlisting}

    import 'dart:async';
    import 'package:bloc/bloc.dart';
    import 'package:meta/meta.dart';
    
    import '../../../../core/usecase/usecase.dart';
    import '../../domain/entities/nonogram.dart';
    import '../../domain/usecases/get_online_nonograms_use_case.dart';
    part 'nonolist_event.dart';
    part 'nonolist_state.dart';
    
    class NonoListBloc extends Bloc<NonoListEvent, NonoListState> {
      NonoListBloc({required this.getNonoListUseCase}) : super(LoadNonoListEvent());
    
      final GetNonoListUseCase getNonoListUseCase;
    
      @override
      Stream<NonoListState> mapEventToState(
        NonoListEvent event,
      ) async* {

        if (event is LoadNonoListEvent) {
          final data = await getNonoListUseCase.call(event.uid);
            yield data.fold(
              (left) => FailedNonoListState(errorMessage: left),
              (right) => FailedNonoListState(nonograms: right));
        }
        if (event is RefreshNonoListEvent) {
          bloc.add(LoadNonoListEvent(event.uid));
        }
      }
    }
    \end{lstlisting}

    \section{Código de Capa de Dominio de Niveles En Línea}

    \label{cap:anexo1-3}
    
    Este apartado contiene la codificación de las entidades, repositorios y casos de uso de la 
    funcionalidad \textit{Niveles en Línea}, (\textit{get\_nonolist\_use\_case.dart}, \textit{nonogram.dart} y \textit{nonolist\_repository.dart}).
    
    \begin{lstlisting}
    
    import 'package:dartz/dartz.dart';

    import '../../../../core/error/failures.dart';
    import '../../../../core/usecase/usecase.dart';
    import '../entities/nonogram.dart';
    import '../repositories/nonolist_repository.dart';

    class GetNonoListUseCase extends UseCase<List<Nonogram>, 
      GetNonoListUseCaseParams> {
      final NonoListRepository repository;

      GetNonoListUseCase({required this.repository});
      @override
      Future<Either<Failure, List<Nonogram>>> call
      (GetNonoListUseCaseParams params) async {
        return await repository.getNonoList();
      }
    }

    class GetNonoListUseCaseParams extends Equatable {
      final String uid;

      GetNonoListUseCaseParams({required this.uid});

      @override
      List<Object?> get props => [uid];
    }

    \end{lstlisting}

    \begin{lstlisting}
    
    import 'package:dartz/dartz.dart';

    import '../../../../core/error/failures.dart';
    import '../entities/nonogram.dart';
    
    abstract class NonoListRepository {
      Future<Either<Failure, List<Nonogram>>> getNonoList();
    }    
    \end{lstlisting}

    \pagebreak

    \begin{lstlisting}

    class Nonogram extends Equatable {
      Nonogram({this.id,
        this.solved,
        this.name,
        this.publishDate,
        this.userId,
        this.figure, 
        this.width, 
        this.height, 
        this.correctTiles});
      int? id;
      bool? solved;
      String? name;
      DateTime? publishDate;
      String? userId;
      String? figure;
      String? width;
      String? height;
      List<int>? correctTiles;

      @override
      List<Object?> get props => [id, solved, name, publishDate, userId,
      figure, width, height, correctTiles];
    }
    \end{lstlisting}

    \section{Código de Capa de Datos de Niveles En Línea}

    \label{cap:anexo1-4}
    
    Este apartado contiene la codificación de las entidades, repositorios y casos de uso de la 
    funcionalidad \textit{Niveles en Línea}, (\textit{nonolist\_repo\_impl.dart, nonogram\_model.dart}
    \linebreak y \textit{nonolist\_data\_source.dart}).
    
    \begin{lstlisting}
      
    import 'package:dartz/dartz.dart';
    import '../../../../core/error/exceptions.dart';
    import '../../../../core/error/failures.dart';
    import '../../domain/repositories/nonolist_repo_impl.dart';
    import '../model/nonogram_model.dart';
    import '../datasources/nonolist_data_source.dart';
    
    class NonoListRepositoryImpl implements NonoListRepository {
      final NonoListDataSource nonoListDataSource;
      final AuthFirebaseDataSource authDataSource;

      NonoListRepositoryImpl({required this.nonoListDataSource, 
          required this.authDataSource});
      @override
      Future<Either<Failure, List<Nonogram>>> getNonoList() async {
        var authFirebaseInstance;
        try {
          authFirebaseInstance = await authDataSource.getAuthInstace();
        } catch (e) {throw Left(AuthFailure)}
        try {
          final remoteNonoList = 
          await authFirebaseInstance.getRemoteNonolist(
            authFirebaseInstance);
          return Right(remoteNonoList);
        } on ServerException { throw Left(ServerFailure);}}}
    \end{lstlisting}

    \begin{lstlisting}

    class NonogramModel extends Nonogram {
      Nonogram({this.id,
        this.solved,
        this.name,
        this.userId,
        this.publishDate,
        this.figure, 
        this.width, 
        this.height, 
        this.correctTiles}):super(
          id: id,
          solved: solved,
          publishDate: publishDate,
          userId: userId,
          figure: figure,
          width: width,
          height: height,
          correctTiles: correctTiles
        )}
    DocumentReference get ref =>
      FirebaseFirestore.instance.collection('nonograms').doc(id);
      
    static Nonogram fromDoc(DocumentSnapshot data) =>
      Nonogram.createFromData(data.data()!, data.id);

    static Future<QuerySnapshot> getNonoListSorted() {
      return FirebaseFirestore.instance
          .collection('nonograms')
          .orderBy('publish_date')
          .get(dataSource);}
    \end{lstlisting}

    \begin{lstlisting}

    import 'dart:convert';
    
    import '../../../../api/endpoints.dart';
    import '../../../../api/api_configuration.dart';
    import '../../../../core/error/exceptions.dart';
    import '../models/nonogram_model.dart';
    
    abstract class NonoListDataSource {
      Future<NonogramModel> getRemoteNonolist(FirebaseFirestore instance);
    }

    class NonoListDataSourceImpl implements NonoListDataSource {
      @override
      Future<bool> getRemoteNonolist() async {
        final nonolist = 
        await instance
          .collection(NonoCollections.nonograms) //<- "nonograms"
          .get()
        if(nonolist != null) {
          return nonolist;
        } else {
          throw ServerException();
        }
      }
    }
  \end{lstlisting}

  \section{Código Inyector de dependencias}

  \label{cap:anexo1-5}
    
  Este último apartado refleja el \textit{inyector de dependencias} del proyecto,
  mediante la librería externa \textit{get\_it}, donde se ``localizan'' todas
  las capas  (\textit{injection\_container.dart}).

\begin{lstlisting}

  final sl = GetIt.instance;

  Future<void> init() async {
    final sharedPreferences = await SharedPreferences.getInstance();
    final auth = await FirebaseAuth.instance;
    final fireStore = await FirebaseFirestore.instance;
  
    sl
      // External
      ..registerLazySingleton(() => sharedPreferences)
      ..registerLazySingleton(() => auth)
      ..registerLazySingleton(() => fireStore)
  
      // BLoC
      ..registerFactory(() => NonoListBloc(getNonoListUseCase: sl()))
  
      // Use cases
      ..registerLazySingleton(() => GetNonoListUseCase(repository: sl()))

      // Repository
      ..registerLazySingleton<NonoListRepository>(() =>
          NonoListRepositoryImpl(
              nonoListDataSource: sl(),
              authDataSource: sl()))
      
      // Data sources
      ..registerLazySingleton<NonoListDataSource>(
          () => NonoListDataSourceImpl())
  }
  
\end{lstlisting}